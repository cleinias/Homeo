%% LyX 2.3.6.1 created this file.  For more info, see http://www.lyx.org/.
%% Do not edit unless you really know what you are doing.
\documentclass[letterpaper,english]{scrartcl}
\usepackage[osf]{mathpazo}
\usepackage{helvet}
\renewcommand{\ttdefault}{cmtl}
\usepackage[T1]{fontenc}
\usepackage[latin9]{luainputenc}
\setcounter{secnumdepth}{5}
\setcounter{tocdepth}{5}
\usepackage{color}
\definecolor{note_fontcolor}{rgb}{0.800781, 0.800781, 0.800781}
\usepackage{verbatim}
\usepackage{amsmath}
\usepackage{setspace}
\usepackage{microtype}
\doublespacing

\makeatletter

%%%%%%%%%%%%%%%%%%%%%%%%%%%%%% LyX specific LaTeX commands.
% Backwards compatibility for LuaTeX < 0.90
\@ifundefined{pageheight}{\let\pageheight\pdfpageheight}{}
\@ifundefined{pagewidth}{\let\pagewidth\pdfpagewidth}{}
\pageheight\paperheight
\pagewidth\paperwidth

%% The greyedout annotation environment
\newenvironment{lyxgreyedout}
  {\textcolor{note_fontcolor}\bgroup\ignorespaces}
  {\ignorespacesafterend\egroup}

%%%%%%%%%%%%%%%%%%%%%%%%%%%%%% User specified LaTeX commands.
% Use the mionoPro fonts. See MinionPro.pdf for the options
%\usepackage[fullfamily,opticals,normalsize,footnotefigures]{MinionPro}

%The following package is needed to have interrupted lists
\usepackage{mdwlist}

% adds microtypography support 

\usepackage{enumitem}
%this is for eliminating rows b/t items with setlist nolistsep
\setlist{nolistsep}  

% For a smaller caption in float figures
\addtokomafont{caption}{\small}

% Koma command to reduce the caption indent in figures
\setcapindent{1em}

\usepackage{colortbl} 
\usepackage{tabularx}
\usepackage[greek, italian, german, french, english]{babel}
\languageattribute{greek}{polutoniko} 
\usepackage[T1]{fontenc}

% Jurabib setup for old-fashioned citations-in-footnote style
% Uncomment AND 
% Do not forget to select the jox style

%\usepackage[oxford=true]{jurabib}
%\jurabibsetup{
%     authorformat=and,
%     titleformat=italic,
%     citefull=first
%}
% \renewcommand{\bibapifont}[1]{``#1''}


% Jurabib setup for Harvard style

\usepackage{jurabib}
\jurabibsetup{authorformat=smallcaps,authorformat=year,round}
\jbyearaftertitle
% In the next line, add a comma or a semicolon just after #1 
% to obtain the classical separation between date and page number
\renewcommand{\jbcitationyearformat}[1]{\unskip,\space#1}


% To use Jurabib idem or ibidem uncomment next line
\jbuseidemhrule

% To convert all endnotes to endnotes uncomment the next TWO lines
%\usepackage{endnotes}
%\let\footnote=\endnote

% To eliminate the date on first page uncomment the next line
%\date{}

% To eliminate the final bibliography replace \thebibliography
% with \nobibliography (and the correct arguments) at the end of 
% the file. You will have to do it on the LaTeX source, though

% The following allows to change the paragraph spacing
% in the printout without changing the screen spacing
% Note: the value set here will override (not be added to) 
% LyX's settings in  Layout>Document>Layout
%
\usepackage{setspace}
\onehalfspace
 
% The next bit shows the comment environment inside a grey box
% Comment it out for final print 
% show my Comments?
\RequirePackage{colortbl, tabularx}
\renewenvironment{comment}
{% replaces \begin{comment}
\noindent
\tabularx{\textwidth}{|>{\columncolor[gray]{0.9}}X|}
\hline
\emph{\textbf{Comment:}}
}
{% replaces \end{comment}
\endtabularx\hrule
}
%
% End of comment printing bit
% 

%
% This bit changes the heading numbering:
%
% It removes the chapter number and it removes all the dots after
% the subsection numbers. It produces numbers "Wittgenstein style":
% for instance the first section of chapter 2 becomes 1 and not 2.1 as 
% standard LaTeX wouldl have it. In general, we have numbering of the
% form 1.2345 instead of 1.1.2.3.4.5 
%
\renewcommand\thesection{\arabic{section}}
\renewcommand\thesubsection{\thesection.\arabic{subsection}}
\renewcommand\thesubsubsection{\thesubsection\arabic{subsubsection}}
\renewcommand\theparagraph{\thesubsubsection\arabic{paragraph}}
\renewcommand\thesubparagraph{\theparagraph\arabic{subparagraph}}

%The following command is needed when working in "outline mode", i.e. when 
% the documents has a long sequence of headings only (with no text in between)
% that would run off the bottom of the page in the pdf file (due to
% LaTeX rule of not allowing a page break directly after a heading.)
% Comment it out for final printings.

 \renewcommand\@afterheading{} 

%\AtBeginDocument{\selectlanguage{english}}

\makeatother

\usepackage{babel}
\begin{document}
\title{Homeostat's units and CTRNN's neurons}

\maketitle
\begin{comment}
Brief comparison between Ashby's and Beer's notations
\end{comment}


\subsection{State equations of a CTRNN unit}

Canonical state equations of a CTRNN's neuron in a network of $N$
elements are \cite{BeerGallagher1992}, \cite{DiPaolo2000}:

\begin{equation}
\tau_{i}\dot{y_{i}}=-y_{i}+\sum_{j=1}^{N}w_{ji}z_{j}+I_{i}\label{eq:CTRNN-unit-general-eq}
\end{equation}

\begin{equation}
z_{j}=\frac{1}{1+e^{b_{j}-y_{j}}}\label{eq:CTRNN-unit-logistic-eq}
\end{equation}

where $\tau$ is the time constant ($=RC$ in the equivalent circuit)
that allows for temporal summation of the incoming input, $y_{i}$represents
the cell potential of neuron $y$, or the \emph{value} of the neuron,
$w_{ji}$ is the weight of the connection from neuron $j$ to neuron
$i$, and $I_{i}$ is the incoming input, which is set to $0$ for
all non input neurons. The $z$ function is the ordinary sigmoid function
that converts the potential of the neuron into a value in the interval
$[0,1]$ as a function of the neuron's potential, $y$, and the neuron's
bias, $b$. The bias term is the parameter in the neuron's state equation
that allows for spatial summation of inputs.\footnote{This equation was originally proposes by \cite{Hop84a}, on the basis
of an analogy between neural networks and $RC$ electronic circuits,
and expressed in terms of resistance, capacitance, and voltage potential,
input current, ($R,C,V,I$) for a neuron $i$ as: $C_{i}\dot{u_{i}}=-\frac{u_{i}}{R_{i}}+\sum_{j}T_{ij}V_{j}+I_{i}$,
where the summative term represents the weighted inputs to neuron
$i$, the second term represents transmembrane resistance, and the
final term, $I_{i}$, is the externally applied input to the neuron.
Normalizing with $\tau_{i}=R_{i}C_{i}$ leads to eq. \eqref{eq:CTRNN-unit-general-eq}.

See \cite[:30]{Beer1990}. Although this formulation has now become
canonical in the evolutionary robotics literature, Beer's original
formulas were more complex and included additional parameters describing
the intrinsic currents generated within a neuron as well as the leak
currents flowing out of it (\cite[:51]{Beer1990}). As these complications
were ignored by the subsequent literature, we will not discuss them
here, although it should not be exceedingly difficult to incorporate
them in the model, especially because Beer, following \cite{Hop84a},
models biological neurons on the basis of electronic circuits.} 

\subsection{State equation of a Homeostat unit}

Ashby describes the behavior of the 4-unit Homeostat as follows in
\emph{Design of a Brain}:\footnote{\cite[ p. 247, 19/12]{Ashby1960}}

\[
\frac{dx_{i}}{dt}=\dot{x_{i}}
\]

\begin{equation}
\frac{d}{dt}(m\dot{x})=-k\dot{x_{i}}+l(p-q)(a_{i1}x_{1}+...+a_{i4}x_{4})\label{eq:Ashby-unit-eq}
\end{equation}

where $x_{i}$ is the $i$ unit's needle's angle of deviation, $m$
is its moment of inertia, $p$ and $q$ are the potentials at the
opposite ends of the fluid-filled trough, $k$ represents the frictional
forces acting on the needle's movement, and $l$ ``depends on the
valve,'' i.e. it is a gain parameter. Rewriting \eqref{eq:Ashby-unit-eq}
in notation closer to CTRNN's would give us, for a Homeostat of $N$
units:

\begin{equation}
m\ddot{x}_{i}=-k\dot{x}_{i}+l(p-q)\sum_{j=1}^{N}a_{ij}x_{j}\label{eq:Unit-CTRNN-not}
\end{equation}

In Ashby's device, the gain parameter, $l$, and the difference in
potential, $p-q$, are the same for all units,\footnote{Or so it seems from Ashby's discussions of his device in all published
descriptions he gave. At any rate, they are never mentioned in the
concrete experiments carried out with the homeostat.} thus allowing to slightly simplify the equation by assuming a constant
gain factor $g=l(p-q)$:

\begin{equation}
m\ddot{x}_{i}=-k\dot{x}_{i}+g\sum a_{ij}x_{j}\label{eq:Unit-CTRNN-not-with-constant}
\end{equation}

\begin{lyxgreyedout}
My implementation substantially follows equation \ref{eq:Unit-CTRNN-not},
with the difference that it does \emph{not }assume $l$, $p$, and
$q$ to be the same across all units. %
\end{lyxgreyedout}

For a slightly different take on the issue: notice that the homeostat's
units cold also be seen as oscillators. Start with a single unit.
If we provide a negative self-connection, and assume no frictional
forces (i.e. the viscosity of the fluid in the trough is 0), then
$-k\dot{x}$ in \ref{eq:Unit-CTRNN-not-with-constant} becomes 0 and
$g\sum a_{ij}x_{i}$ becomes simply $-ga_{jj}x$, with $a_{ii}$being
the weight of the self-connection. Putting $ga_{ii=}k$, the unit's
behavior is described by 

\[
m\ddot{x}=-kx
\]

which is the canonical equation for a harmonic oscillator. If we add
a viscosity factor $v$ due to the trough's fluid and assume it proportional
to velocity, we obtain the canonical equation for a damped oscillator:
$m\ddot{x}=-v\dot{x}-kx$ . Thus, the basic building block of a Homeostat
is a damped harmonic oscillator. And a Homeostat (leaving uniselectors'
action aside) is simply a collection of variously coupled harmonic
oscillators. 

On the other hand, in a ``real-world'' translation of the homeostat
we can assume that the real world input the unit are connected to
are not self-regulating, i.e. they are not (damped) harmonic oscillators.
It follows that a \emph{single} negatively self-connected unit connected
to the ``real'' environment is equivalent to a driven harmonic oscillator:
\[
m\ddot{x}=-v\dot{x}-kx+aF(t)
\]

where $F(t)$ and $a$ are, respectively, the input the unit is receiving
from the environment and the weight it assigns to that input. A ``real
world'' Homeostat (or its equivalent homeostat simulation, if possible)
is thus a collection of driven harmonic oscillators (uniselectors'
actions aside).

\subsection{Comparison between the two}

The comparison between\eqref{eq:CTRNN-unit-general-eq} and \eqref{eq:Unit-CTRNN-not}
can be carried out at three levels: the general shape of the description
(the model), the terms involved, and the relation between a unit's
value (i.e. potential and rotation angle, resp., for CTRNNs and Ashby)
and a unit's output. 

\subsubsection{The mathematical model}

Here is where we find the greatest difference between the two. Although
the models are remarkably similar, the important difference is that
the value $y$ (potential) of CTRNN unit is defined in terms of its
\emph{first} derivative, whereas the value $x$ of a Ashbian unit
is expressed in term of its \emph{second }and first derivatives. 

Mechanically interpreted in traditional notation: $F=ma$ or $m\dot{v}=F$,
and $v=\dot{s}$, with the traditional abbreviations for mass, acceleration,
velocity and displacement, where the force $F$ acting on the object
can be further decomposed in a reactive frictional force proportional
to velocity plus the active forces acting on the object (assuming,
that is, the simplest mathematical model of friction). Thus the unit
is fundamentally modeled as a Newtonian object rotating on a friction-affected
pivot, and the basic parameters affecting its value (angular position)
are its mass (or moment of inertia) and the friction it exerts (assuming
the difference in potential is a constant; otherwise it would have
one additional parameter): $m,k,(p-q)$. If we try to interpret the
CTRNN unit mechanically, for the sake of comparing it to the Homeostat,
we see that \eqref{eq:CTRNN-unit-general-eq} can be read as the model
of an \emph{Aristotelian }object, that is, an object whose velocity
is proportional to the forces it is subjected to. In other words,
the CTRNN unit is normally at rest in absence of incoming forces (its
biological equivalent, the real neuron having an appropriately called
resting potential), whereas a Homeostat's unit in the same situation
is in uniform motion. 

\begin{comment}
Not really sure about the preceding point. Here is a tentative but
perhaps better discussion. Notice, however, that there is really no
notion of inertial mass in Aristotelian physics, so the interpretation
of $C_{i}$ as such is not quite correct.
\end{comment}

The problem is that a straightforward ``Aristotelian'' dynamical
model would have velocity equal to force over mass ($F=mv$ or, rather,
$v=\frac{F}{m}$, in differential term: $\dot{ms}=F$). Now, assuming,
in the CTRNN case, that the forces are equal to the incoming inputs
plus the externally applied current, and using $y$ to represent the
mechanical displacement, an Aristotelian model would give us the following
equation: $m\dot{y}=\sum_{j}w_{ij}y_{j}+I_{i}$. Comparing this equation
with eq. \eqref{eq:CTRNN-unit-general-eq} makes the problem clear:
we are missing a term. Namely, $-y_{i}$. If we rewrite the equation
according to its original formulation, though, namely as $C_{i}\dot{y_{i}}=-\frac{y_{i}}{R_{i}}+\sum_{j}w_{ij}y_{j}+I_{i}$,
we can perhaps make some progress. Now this shows, interpreted mechanically,
that the applied force has three components, one of which is proportional
to displacement $y$ and to a resistive factor $R$, and the other
two are dependent on the directly applied forces. This is not quite
the original Aristotelian dynamical model, which explains friction
as proportional to mass (and therefore contained in the $C_{i}$ term
in the equation). Furthermore, having a resistive force proportional
to displacement models, in mechanical terms, the resistance encountered
when driving a stake through the ground, i.e. the resistance of an
extended object entering a viscous fluid. I suppose what I am saying
is that the electro-mechanical device that would (approximately) realize
the CTRNN equation and be as close as possible to Ashby's contraption
would be made up of 
\begin{enumerate}
\item highly viscous ``wells'' (with and electric potential between surface
and bottom) with ``stakes'' being driven in and out of them. This
part would implement the $-\frac{y_{i}}{R_{i}}$ term.
\item very heavy friction in the mechanism's joints, surface, etc. This
would approximate the Aristotelian modeling of force being proportional
to velocity and not to acceleration, and thus the $C_{i}\dot{y_{i}}$
component instead of the $m\ddot{y_{i}}$ component in Ashby's model.
\end{enumerate}
Now the second big difference is that there is a linear function (in
fact, the identity function) in Ashby's model, whereas there is a
non-linear function, the logistic one, in CTRNN case.

\begin{comment}
Go on from here
\end{comment}

The main difference, of course, is that whereas Ashby's Units have
an identity function from unit's value to output, CTRNN units have
a sigmoid, hence a non-linear, though continuosly differentiable function.
The reason behind the adoption of a sigmoid function nonlinear function
is due to the wish to model the frequency coding properties of real
neurons, since such a function converts, over a short time span, a
neuron's value into a firing frequency. 

\begin{comment}
Say something on frequency coding for real value as the job being
performed here by the sigmoid function.

Notice, BTW, that Ashby is perfectly aware of the spiking, frequency
coding capacities of the nervous system (obviously). See for instance
\cite[ p. 57]{Ashby1947}.
\end{comment}


\subsubsection{Term to term}

Now for the equivalence with Ashby's Homeostat. Its units are electro-mechanical
contraptions where the values of interest to the modeler are the physical
properties of the constantly moving parts and the forces acting on
them. The state of a unit is thus represented by the velocity of the
needle, the mechanical force acting on it (that is, the torque produced
by the magnets coiled around its pivot as a result of the incoming
currents from the other units), the mass of the needle, and the resistance
the needle encounters in its movement through the fluid-filled trough.
A term by term equivalence is can perhaps be found by interpreting
the capacitance $C_{i}$ and resistance $R_{i}$ of the CTRNN model
as being, roughly, functional equivalent to, respectively, mass and
friction in Ashby's model. 

\begin{comment}
Eliminate the stuff that follows this comment until the end of this
subsubsection
\end{comment}

In mechanical terms, since the decay constant $\tau$ represents the
neuron's resistance to a change of state, it is equivalent to needle's
mass providing an inertial resistance against change of velocity.
The bias term $b$ represents a neuron's resistance against firing
and is equivalent to the the frictional resistive forces acting against
the needle's displacement. The weights of the incoming connections,
$w_{ji},$ are obviously equivalent to the various resistors values
on the incoming currents, while the incoming current, $I_{i},$ is
normally equal to $0$ unless the Homeostat's operator decides to
manually manipulate one or more of the units (as in Ashby's various
experiments), in which case it corresponds to the external force acting
on the needle. 

\subsection{Conclusion}

Since in the original Homeostat all units are constructionally identical,
the only free parameters governing the evolution of the network are
the weights $w_{ij}$ controlling the connections between the units.
If, on the other hand, we consider the possibility of a ``generalized''
Homeostat (rather easily realizable through a computer simulation,
continuous or discrete), we should consider all the parameters in
eq. \eqref{eq:Unit-CTRNN-not} as being independent, as we can easily
construct units with different masses, different viscosities, potentials,
gains, etc. That means that the simulation (and training, perhaps
via GA techniques) of a homeostat of $j$ units would entail the determination
of $4$ parameters per unit (mass, viscosity, gain and potential)
+ $j$ weights, totaling $j(j+4)$ networks parameters. This number
is actually not very different from the CTRNN standard of $j(j+2)$
(weight, plus time constant and bias). 

\begin{comment}
Add something on the lack of plausibility of Ashby's units as neuron
(even highly simplified), and their more likely identification as
higher level functional units with possibly little connection to the
biological details. In other words, suggest that a naturalistic interpretation,
or, even a mimetic, interpretation of the Homeostat is not plausible.
See whether the dates for the early $RC$ models of the neuron (Hodgkin?)
overlap with the composition dates of \emph{DofB} 
\end{comment}

\bibliographystyle{jurabib}
\bibliography{/home/stefano/Documents/Projects/Homeostat/Biblios/Homeostat-Jurabib,/home/stefano/Documents/Projects/Homeostat/Biblios/Recurrent-Neural-Network-Biblios-Lee-Giles-up-to-1991}

\end{document}
