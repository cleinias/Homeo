%% LyX 2.0.5 created this file.  For more info, see http://www.lyx.org/.
%% Do not edit unless you really know what you are doing.
\documentclass[french,ngerman,english]{scrartcl}
\usepackage[T1]{fontenc}
\usepackage[latin9]{inputenc}

\makeatletter
%%%%%%%%%%%%%%%%%%%%%%%%%%%%%% User specified LaTeX commands.
\usepackage{colortbl} 
\usepackage{tabularx}
\usepackage[greek, italian, german, french, english]{babel}
\languageattribute{greek}{polutoniko} 
\usepackage[T1]{fontenc}
\usepackage[oxford=true]{jurabib}
\jurabibsetup{
     authorformat=and,
     titleformat=italic,
     citefull=first
}
\renewcommand{\bibapifont}[1]{``#1''}

% To use Jurabib idem or ibidem uncomment next line
%\jbuseidemhrule

% To convert all endnotes to endnotes uncomment the next TWO lines
%\usepackage{endnotes}
%\let\footnote=\endnote

% To eliminate the date on first page uncomment the next line
%\date{}

% To eliminate the final bibliography replace \thebibliography
% with \nobibliography (and the correct arguments) at the end of 
% the file. You will have to do it on the LaTeX source, though

% The following allows to change the paragraph spacing
% in the printout without changing the screen spacing
% Note: the value set here will override (not be added to) 
% LyX's settings in  Layout>Document>Layout
%
%\usepackage{setspace}
%\onehalfspace
 
% The next bit shows the comment environment inside a grey box
% Comment it out for final print 
% show my Comments?
%\RequirePackage{colortbl, tabularx}
%\renewenvironment{comment}
%{% replaces \begin{comment}
%\noindent
%\tabularx{\textwidth}{|>{\columncolor[gray]{0.9}}X|}
%\hline
%\emph{\textbf{Comment:}}
%}
%{% replaces \end{comment}
%\endtabularx\hrule
%}
%
% End of comment printing bit
% 

%
% This bit changes the heading numbering:
%
% It removes the chapter number and it removes all the dots after
% the subsection numbers. It produces numbers "Wittgenstein style":
% for instance the first section of chapter 2 becomes 1 and not 2.1 as 
% standard LaTeX wouldl have it. In general, we have numbering of the
% form 1.2345 instead of 1.1.2.3.4.5 
%
\renewcommand\thesection{\arabic{section}}
\renewcommand\thesubsection{\thesection.\arabic{subsection}}
\renewcommand\thesubsubsection{\thesubsection\arabic{subsubsection}}
\renewcommand\theparagraph{\thesubsubsection\arabic{paragraph}}
\renewcommand\thesubparagraph{\theparagraph\arabic{subparagraph}}
\AtBeginDocument{\selectlanguage{english}}

\makeatother

\usepackage{babel}
\makeatletter
\addto\extrasfrench{%
   \providecommand{\og}{\leavevmode\flqq~}%
   \providecommand{\fg}{\ifdim\lastskip>\z@\unskip\fi~\frqq}%
}

\makeatother
\begin{document}

\title{Notes on various lines of inquiry (and deliverables) on the Homeostat
project (with tasks when appropriate).}


\author{Stefano Franchi}

\maketitle

\section{General}

This documents lists the various lines of inquiry possibly to be pursued
in the Homeostat project. It mainly serves as a reminder to myself
of the various open possibilities, as well as a document tracking
the open tasks and the scheduling of jobs.


\section{Specific lines of inquiry}


\subsection{Main philosophical point}


\subsubsection{Passivity versus activity as the basic interpretation of the Homeostat's
operations}


\paragraph{Various points to consider }
\begin{itemize}
\item Freud's interpretation (\foreignlanguage{ngerman}{\emph{Triebe}}\emph{
and }\foreignlanguage{ngerman}{\emph{Triebeschicksalen}}) of drives
(``\foreignlanguage{french}{pulsions}'') and the psychic apparatus's
function as minimizing the incoming stimuli (the \emph{constancy }principle,
to see in the context of the pleasure principle) DONE
\item Spinoza's doctrine of the \emph{conatus}, recently reinterpreted (DiPaolo
following Varela), as an alternative viewpoint to be considered and
rejected.
\item Jonas's criticism of classic cybernetics, and Varela's recover of
a different kind of cybernetics (Ashby/British inspired) as escaping
Jonas's criticism and being instead aligned with his philosophy of
biology (i.e. teleologism) DONE
\item Contingency as the twin brother of passivity (the passive is contingent,
the active is teleological)
\end{itemize}

\subsection{Technical/Simulations}
\begin{itemize}
\item Redo (my own redo of) Ashby's seven experiments on the basis of the
newly redesigned Homeo package (with Newtonian units, etc).
\item Some experiments on connectivity on large-scale homeostats with $n$
units to test Ashby's original hypothesis about low connectivity as
necessary to stability (and necessarily mediated by the environment).
\item Experiments on ``biological'' simulations with the homeostat. This
will require input and/or output (motor) units. Thus, either a simulated
environment (explorations needed) or an embedded Homeo package.
\end{itemize}

\subsection{\label{sub:Theoretical-explorations}Theoretical explorations}

Following the notes of 3/22-26/08, explore the Homeostat in more general
term as a ``differential machine'' of sorts, i.e. as a simulator
of various families of biological simulators distinguished by the
differential equation descriptions of the basic unit.


\section{Papers}

This are very preliminary ideas on 4 papers to write on the Homeo
experience (or perhaps it is 4 categories of papers, with at least
(and at most?) one instance per class)


\subsection{A semi-technical paper (\emph{Constructivist Foundations}) describing:}
\begin{itemize}
\item The reasons suggesting the development of a homeostat general simulator
(i.e. the Ashby's renaissance, the strange convergence between very
different strands of philosophy (Jonas's Phenomenology, naturalistic-inspired
Phil of mind, AI, etc) around the Homeostat, as well as the generally
``shallow'' (from my point of view, of course) \emph{overall} discussions
of the Homeostat
\item The package itself, emphasizing its very general approach as prerequisite
to a broad range of experiments that start from Ashby's original experiences,
but are meant to go beyond them
\item The results of the replication of Ashby's experiments, perhaps a first
sample of the DiPaolo-like experiments
\item A conclusion on further work
\end{itemize}

\subsection{A Non-Technical, but ``heavy'' philosophical paper (for \emph{Archeo
of AI}) \emph{DONE}}
\begin{itemize}
\item This is a paper expanding in much greater detail the first of the
bulleted item in the semi-technical paper described above. The main
focus is on Varela/DiPaolo revival of the Homeostat in the context
of Jonas's philosophy of biology vs. a different interpretation of
the Homeostat as a fundamentally passive machine (a \emph{machina
sopora}) that actually forces a deep reconceptualization of the conception
of passivity and the associate notion of the contingent
\end{itemize}

\subsection{An application and/or study of the theses above to a real environment}
\begin{itemize}
\item The move here is from a general discussion limited, at best, to Ashby's
experiments as example of practical implementation, to a more concrete
set of tests in ``real'' environments. The real environments could
actually be ``real real'' (requiring real robots) or ``real simulated''
(requiring a simulation platform)
\end{itemize}

\subsection{A discussion of a generalized Homeo platform for the exploration
of various homeostatic ``implementations''}
\begin{itemize}
\item See \ref{sub:Theoretical-explorations} above. This of course can
only be done after the actual explorations have been carried out.
\end{itemize}

\section{Details}


\section{Work to do}


\subsection{General}
\begin{enumerate}
\item Complete unit testing of Homeo Package
\item Carry out Ashby experiments
\item Explore robotic simulation packages DONE
\item Explore khepera packages DONE\end{enumerate}

\end{document}
