\documentclass[11pt,a4paper]{article}

\usepackage{fontspec}
\setmainfont{Minion Pro}

\usepackage{booktabs}
\usepackage{longtable}
\usepackage{array}
\usepackage{enumitem}
\usepackage{fancyvrb}
\usepackage{xcolor}
\usepackage[colorlinks=true, linkcolor=blue!60!black, urlcolor=blue!60!black]{hyperref}
\usepackage[margin=2.5cm]{geometry}
\usepackage{parskip}

% Code listing style
\definecolor{codebg}{gray}{0.95}
\usepackage{listings}
\lstset{
  language=Python,
  basicstyle=\ttfamily\small,
  backgroundcolor=\color{codebg},
  frame=single,
  framerule=0pt,
  xleftmargin=1em,
  xrightmargin=1em,
  breaklines=true,
  columns=fullflexible,
  keepspaces=true,
  showstringspaces=false,
  keywordstyle=\bfseries,
  commentstyle=\itshape\color{gray},
}

\title{\bfseries Hunger-Driven Chemotaxis\\[0.3em]
  \large Implementation Plan}
\author{}
\date{}

\begin{document}
\maketitle
\thispagestyle{empty}

\section{Overview}

The experiment extends the existing phototaxis setup with an internal \textbf{battery} that depletes over time and recharges when the robot is near the light source. A new \textbf{interoceptive HomeoUnit} monitors battery level, acting as a ``hunger drive'' that influences motor behaviour through the homeostat's own homeostatic dynamics.

\section{Architecture Mapping}

\begin{center}
\begin{tabular}{@{}p{4cm}p{9.5cm}@{}}
\toprule
\textbf{Concept} & \textbf{Implementation} \\
\midrule
Robot with sensors/motors & Already exists: \texttt{KheperaRobot} + \texttt{HOMEO\_DiffMotor}\,/\,\texttt{HOMEO\_LightSensor} transducers \\[0.5em]
Battery that discharges & \textbf{New}: \texttt{Battery} object in \texttt{KheperaSimulator}, tracked as a body/state in the simulation \\[0.5em]
Interoceptive ``hunger'' sensor & \textbf{New}: \texttt{HOMEO\_BatterySensor} transducer + a \texttt{HomeoUnitInput} reading battery level \\[0.5em]
Battery recharges near light & \textbf{New}: logic in \texttt{advanceSim()} that recharges battery proportionally to light irradiance at robot position \\
\bottomrule
\end{tabular}
\end{center}

\section{Step-by-step Plan}

\subsection{Step 1: Add Battery to KheperaSimulator}

Add a simple battery model to \texttt{KheperaSimulator.py}:

\begin{lstlisting}
class KheperaBattery:
    def __init__(self, capacity=1.0, discharge_rate=0.001,
                 recharge_factor=0.01):
        self.capacity = capacity
        self.level = capacity          # starts full
        self.discharge_rate = discharge_rate
        self.recharge_factor = recharge_factor

    def tick(self, irradiance=0.0):
        """Discharge, then recharge proportionally to
        irradiance at robot's position."""
        self.level -= self.discharge_rate
        self.level += self.recharge_factor * irradiance
        self.level = max(0.0, min(self.capacity, self.level))
        return self.level

    def range(self):
        return (0.0, self.capacity)
\end{lstlisting}

\subsection{Step 2: Wire battery into KheperaRobot and advanceSim}

\begin{itemize}[nosep]
  \item Give \texttt{KheperaRobot} an optional \texttt{battery} attribute (default \texttt{None} for backward compatibility).
  \item In \texttt{KheperaSimulation.advanceSim()}, if the robot has a battery, compute the total irradiance at the robot's current position from all detectable lights, then call \texttt{battery.tick(irradiance)}.
\end{itemize}

The battery discharges every simulation step and recharges only when the robot is close enough to the light to receive non-negligible irradiance---exactly the ``hunger'' dynamic from the research note.

\subsection{Step 3: Add HOMEO\_BatterySensor transducer}

In \texttt{RobotSimulator/Transducer.py}, following the existing \texttt{HOMEO\_LightSensor} pattern:

\begin{lstlisting}
class HOMEO_BatterySensor(Transducer):
    """Interoceptive transducer that reads the
    robot's battery level."""

    def __init__(self, robotRef):
        self.robot = robotRef

    def read(self):
        return self.robot.battery.level

    def range(self):
        return self.robot.battery.range()

    def act(self):
        raise TransducerException(
            "Battery sensor cannot act")
\end{lstlisting}

This bridges the robot's internal state (battery) to the homeostat via a standard \texttt{HomeoUnitInput}, just like \texttt{HOMEO\_LightSensor} bridges external light readings.

\subsection{Step 4: Create world setup function with battery}

In \texttt{KheperaSimulator.py}, add a new world setup method:

\begin{lstlisting}
def kheperaBraitenberg2_HOMEO_World_battery(self):
    """Like kheperaBraitenberg2_HOMEO_World
    but with a battery."""
    world = self.kheperaBraitenberg2_HOMEO_World()
    self.allBodies[self.robotName].battery = \
        KheperaBattery()
    return world
\end{lstlisting}

\subsection{Step 5: Create experiment function in HomeoExperiments.py}

Add a new GA experiment function that builds an \textbf{8-unit homeostat}:

\begin{center}
\begin{tabular}{@{}lll@{}}
\toprule
\textbf{Unit} & \textbf{Type} & \textbf{Role} \\
\midrule
Left Motor      & \texttt{HomeoUnitNewtonianActuator} & Drives left wheel \\
Right Motor     & \texttt{HomeoUnitNewtonianActuator} & Drives right wheel \\
Left Eye        & \texttt{HomeoUnitNewtonian}          & Processes left eye input \\
Right Eye       & \texttt{HomeoUnitNewtonian}          & Processes right eye input \\
\textbf{Hunger} & \texttt{HomeoUnitNewtonian}          & Processes battery level \\
Left Sensor     & \texttt{HomeoUnitInput}               & Reads left eye irradiance \\
Right Sensor    & \texttt{HomeoUnitInput}               & Reads right eye irradiance \\
\textbf{Battery Sensor} & \texttt{HomeoUnitInput}      & Reads battery level \\
\bottomrule
\end{tabular}
\end{center}

The \textbf{Hunger unit} is a standard \texttt{HomeoUnitNewtonian} that receives input from the Battery Sensor (and potentially from other units). Its essential parameter values are evolved by the GA. When battery is low, the Battery Sensor feeds a high ``hunger'' deviation into the Hunger unit, which propagates through the homeostat's connections to influence motor outputs---the homeostatic dynamics themselves determine \emph{how} hunger affects behaviour.

\textbf{Genome structure}: 5 evolved units $\times$ 4 essential params $+$ 5 evolved units $\times$ 8 total units $=$ \textbf{60 genes} (up from 40 in the standard phototaxis experiment).

\subsection{Step 6: Define fitness function}

Three options:

\begin{description}[style=nextline]
  \item[Option A --- Battery survival] Fitness = number of ticks the battery remains above~0 (or above some threshold).
  \item[Option B --- Average battery level] Robots that stay near the light maintain high battery over the run.
  \item[Option C --- Combined] Final distance to target $\times$ battery penalty.
\end{description}

A new method on the backend (e.g., \texttt{averageBatteryLevel()}) would report the battery metric.

\subsection{Step 7: Lesion studies (future)}

Once a successful controller is evolved, systematic lesion studies would remove the Hunger unit or lesion specific connections (e.g., Hunger$\to$Motor) and compare performance. These don't require new code---just experiment variants that disable specific connections after loading an evolved genome.

\section{Key Design Decisions}

\begin{enumerate}
  \item \textbf{How many units?} 8-unit homeostat with distinct interoceptive channel, or feed battery into existing eye units (losing the separate ``interoceptive'' channel)?

  \item \textbf{Fitness measure}: Battery survival time vs.\ average battery level vs.\ distance-based?

  \item \textbf{Battery parameters}: Discharge rate and recharge factor determine task difficulty. Fixed or included in the GA?

  \item \textbf{Genome size}: 60 genes (from 40) is manageable but increases search space---may need larger population or more generations.
\end{enumerate}

\section{Files to Modify/Create}

\begin{center}
\small
\begin{tabular}{@{}p{5.5cm}p{8cm}@{}}
\toprule
\textbf{File} & \textbf{Change} \\
\midrule
\texttt{KheperaSimulator.py} & Add \texttt{KheperaBattery} class; wire into \texttt{KheperaRobot} and \texttt{advanceSim}; add battery world setup \\[0.3em]
\texttt{Transducer.py} & Add \texttt{HOMEO\_BatterySensor} class \\[0.3em]
\texttt{HomeoExperiments.py} & Add hunger-phototaxis experiment function \\[0.3em]
\texttt{HomeoGenAlgGui.py} & Add experiment to combo box; possibly add battery-based fitness path \\[0.3em]
\texttt{SimulatorBackend.py} & Add \texttt{averageBatteryLevel()} or similar method \\[0.3em]
\texttt{HomeoGATest.py} & Tests for new experiment attributes \\
\bottomrule
\end{tabular}
\end{center}

\end{document}
