%% LyX 2.4.4 created this file.  For more info, see https://www.lyx.org/.
%% Do not edit unless you really know what you are doing.
\documentclass[oneside,english,article, 11pt]{memoir}
\usepackage[T1]{fontenc}
\setcounter{secnumdepth}{3}
\setcounter{tocdepth}{3}
\usepackage{color}
\definecolor{note_fontcolor}{rgb}{0.800781, 0.800781, 0.800781}
\usepackage{amsmath}
\usepackage{amsthm}
\usepackage{amssymb}
\usepackage{graphicx}
\usepackage[authoryear]{natbib}
\OnehalfSpacing

\makeatletter

%%%%%%%%%%%%%%%%%%%%%%%%%%%%%% LyX specific LaTeX commands.
%% The greyedout annotation environment
\newenvironment{lyxgreyedout}
{\normalfont\normalsize\textcolor{note_fontcolor}\bgroup\ignorespaces}
{\ignorespacesafterend\egroup}

%%%%%%%%%%%%%%%%%%%%%%%%%%%%%% Textclass specific LaTeX commands.
\theoremstyle{plain}
\newtheorem{thm}{\protect\theoremname}

%%%%%%%%%%%%%%%%%%%%%%%%%%%%%% User specified LaTeX commands.
\addbibresource{/home/stefano/Documents/Biblios/Italian-philosophy.bib}
\addbibresource{/home/stefano/Documents/Biblios/Stefano-Franchi-Works.bib}
\addbibresource{/home/stefano/Documents/Books/Play-and-Passivity/Dissertation-references.bib}
\addbibresource{/home/stefano/Documents/Biblios/additional-refs-for-Il-Pensiero-Vivente.bib}

\ifdefined\showcaptionsetup
 % Caption package is used. Advise subfig not to load it again.
 \PassOptionsToPackage{caption=false}{subfig}
\fi
\usepackage{subfig}
\makeatother

\usepackage{babel}
\providecommand{\theoremname}{Theorem}

\begin{document}
\title{Converting Homeostat equations into state form}
\maketitle

\chapter{Converting Homeostat's general equation into state form }

The following note tries to convert the general equation of the homeostat
and of a homeostat-driven vehicle into state form in order to carry
out some mathematical analysis on the resulting systems.

\section{Homeostat general equation}

The general equation of a single unit $i$ of a $N$ units homeostat
is:

\begin{equation}
m_{i}\ddot{y_{i}}=-v_{i}\dot{y_{i}}+\sum_{j=1}^{N}w_{ij}y_{j}+c_{i}u_{i}\label{eq:homeo-unit-general}
\end{equation}

where $m_{i}$ is the mass of unit $i$'s needle, $v_{i}$ is the
viscosity of unit $i$'s medium, $w_{ij}$ is the weight of the connection
from unit $j$ to unit $i$, and $u_{i}$is the (possible) input to
unit $i$ (possibly multiplied by a unit-specific coefficient $c_{i}$).
Assuming the state variables, for each unit $i$, are $\begin{bmatrix}x_{i_{1}}\\
x_{i_{2}}
\end{bmatrix}=\begin{bmatrix}y_{i}\\
\dot{y_{i}}
\end{bmatrix}$, that our $y$ (the output) is equal to $y_{i}$ and that the input
is $c,$ we have that $\dot{y}_{i}=\dot{x}_{i_{1}}=x_{i_{2}}$ and
$\ddot{y}_{i}=\dot{x}_{i_{2}}=\frac{1}{m}(-vx_{i_{2}}+\sum_{j=1}^{N}w_{ij}x_{j_{1}}+u_{i})$. 

Now for the state form. We need to convert the above expression into
a general expression of the form $\dot{x}_{i}=Ax+Bu$, $y=Cx$ with
$A,B,C$ as matrices. First, for the order of the matrices: $x$ is
a vector (column matrix) of size $2i\,x\,1$, and $u$ is a vector
of size $2i\,x\,1.$Hence the $A$ matrix will be of size $2i\,x\,2i$,
$B$ will be of size $2i\,x\,1$ and $C$ will be $1\,x\,2i$. Our
final equations of the homeostat in state form, will thus be the following:
\begin{equation}
\dot{x}=\begin{bmatrix}0 & 1 & 0 & 0 & 0 & 0\\
\frac{w_{11}}{m_{1}} & -\frac{v_{1}}{m_{1}} & \frac{w_{12}}{m_{1}} & 0 & \frac{w_{1i}}{m_{1}} & 0\\
... & ... & ... & ... & ... & ...\\
... & ... & ... & ... & ... & ...\\
\frac{w_{1i}}{m_{i}} & 0 & \frac{w_{2i}}{m_{i}} & 0 & \frac{w_{ii}}{m_{i}} & -\frac{v_{i}}{m_{i}}\\
0 & 0 & 0 & 0 & 0 & 1
\end{bmatrix}\begin{bmatrix}x_{1_{1}}\\
x_{1_{2}}\\
...\\
...\\
x_{i_{1}}\\
x_{i_{2}}
\end{bmatrix}+\begin{bmatrix}0\\
\frac{c_{1}}{m_{1}}\\
0\\
...\\
0\\
\frac{c_{i}}{m_{i}}
\end{bmatrix}u\label{eq:H-State-Form-1}
\end{equation}

In other words: every $2i$-long odd-numbered row of the $A$ matrix
will have all zero except for the position $2^{nd}$, $4^{th}$, $6^{th}$,
etc. position, which will contain a $1.$ The even numbered rows of
the $A$ matrix will have the weights divided by the masses in all
the odd-numbered columns, and the (negative) viscosity divided by
the mass in the $2^{nd}$, $4^{th}$, $6^{th}$, etc. position. The
$B$ matrix will have all zeroes in the odd-numbered rows and the
unit's input coefficient divided by the unit's mass in the even-numbered
rows. For the second equation, we will have:

\begin{equation}
y=\begin{bmatrix}1 & 0 & ... & ... & 1 & 0\end{bmatrix}x\label{eq:H-State-Form-2}
\end{equation}

As an example, for a 2-unit homeostat, the matrices would thus be:\medskip{}

\begin{center}
$A=\begin{bmatrix}0 & 1 & 0 & 0\\
\frac{w_{11}}{m_{1}} & -\frac{v_{1}}{m_{1}} & \frac{w_{12}}{m_{1}} & 0\\
0 & 0 & 0 & 1\\
\frac{w_{12}}{m_{2}} & 0 & \frac{w_{22}}{m_{2}} & -\frac{v_{2}}{m_{2}}
\end{bmatrix}$ $B=\begin{bmatrix}0\\
\frac{c_{1}}{m_{1}}\\
0\\
\frac{c_{2}}{m_{2}}
\end{bmatrix}$ $C=\begin{bmatrix}1 & 0 & 1 & 0\end{bmatrix}$
\par\end{center}

and with the default value in Homeo of $m_{1}=m_{2}=100$ and $v_{1}=v_{2}=10$,
$c_{1}=c_{2}=1$ we obtain the matrices:

\begin{center}
$A^{'=}\left[\begin{array}{rrrr}
0 & 1 & 0 & 0\\
\frac{w_{22}}{100} & -\frac{1}{10} & \frac{w_{12}}{100} & 0\\
0 & 0 & 0 & 1\\
\frac{w_{12}}{100} & 0 & \frac{w_{22}}{100} & -\frac{1}{10}
\end{array}\right]$$B^{'}=\begin{bmatrix}0\\
\frac{1}{100}\\
0\\
\frac{1}{100}
\end{bmatrix}$$C^{'}=C=\begin{bmatrix}1 & 0 & 1 & 0\end{bmatrix}$
\par\end{center}

\section{Controllability}

Controllability theorem (see Murray):
\begin{thm}
A system is completely controllable if and only if $rank(\Gamma)=n$,
where $n$ is the dimension of the state variable and $\Gamma=\begin{bmatrix}B & AB & ... & A^{n-1}B\end{bmatrix}$
is the system's controllability matrix
\end{thm}
Let us try for a 2 units homeostat, in which case $A$ and $B$ are
defined as above, $n=4$ and $\Gamma=\begin{bmatrix}B & AB & A^{2}B & A^{3}B\end{bmatrix}.$
The symbolic calculation is rather hard to do by hand, but with Sage
to the rescue we obtain the scary looking matrix:

$\Gamma=\left(\begin{array}{rrrr}
0 & \frac{c_{1}}{m_{1}} & -\frac{c_{1}v_{1}}{m_{1}^{2}} & \frac{c_{1}{\left(\frac{v_{1}^{2}}{m_{1}^{2}}+\frac{w_{11}}{m_{1}}\right)}}{m_{1}}+\frac{c_{2}w_{12}}{m_{1}m_{2}}\\
\frac{c_{1}}{m_{1}} & -\frac{c_{1}v_{1}}{m_{1}^{2}} & \frac{c_{1}{\left(\frac{v_{1}^{2}}{m_{1}^{2}}+\frac{w_{11}}{m_{1}}\right)}}{m_{1}}+\frac{c_{2}w_{12}}{m_{1}m_{2}} & -\frac{c_{1}{\left(\frac{v_{1}{\left(\frac{v_{1}^{2}}{m_{1}^{2}}+\frac{w_{11}}{m_{1}}\right)}}{m_{1}}+\frac{v_{1}w_{11}}{m_{1}^{2}}\right)}}{m_{1}}-\frac{c_{2}{\left(\frac{v_{1}w_{12}}{m_{1}^{2}}+\frac{v_{2}w_{12}}{m_{1}m_{2}}\right)}}{m_{2}}\\
0 & \frac{c_{2}}{m_{2}} & -\frac{c_{2}v_{2}}{m_{2}^{2}} & \frac{c_{2}{\left(\frac{v_{2}^{2}}{m_{2}^{2}}+\frac{w_{22}}{m_{2}}\right)}}{m_{2}}+\frac{c_{1}w_{12}}{m_{1}m_{2}}\\
\frac{c_{2}}{m_{2}} & -\frac{c_{2}v_{2}}{m_{2}^{2}} & \frac{c_{2}{\left(\frac{v_{2}^{2}}{m_{2}^{2}}+\frac{w_{22}}{m_{2}}\right)}}{m_{2}}+\frac{c_{1}w_{12}}{m_{1}m_{2}} & -\frac{c_{2}{\left(\frac{v_{2}{\left(\frac{v_{2}^{2}}{m_{2}^{2}}+\frac{w_{22}}{m_{2}}\right)}}{m_{2}}+\frac{v_{2}w_{22}}{m_{2}^{2}}\right)}}{m_{2}}-\frac{c_{1}{\left(\frac{v_{1}w_{12}}{m_{1}m_{2}}+\frac{v_{2}w_{12}}{m_{2}^{2}}\right)}}{m_{1}}
\end{array}\right)$

\medskip{}

whose rank is indeed 4. Hence the system is controllable, \emph{assuming
all the coefficient are indeed different!} 

Notice that setting the constants $m_{1,}m_{2,}v_{1,}v_{2}$ to their
default values in Homeo of, respectively, $100,100,10,10$ and the
constant $c_{1,}c_{2}$ to $1,$the resulting matrix would be:

\[
\Gamma^{'}=\mbox{\ensuremath{\left[\begin{array}{rrrr}
0 & \frac{1}{100} & -\frac{1}{1000} & \frac{w_{11}}{10000}+\frac{w_{12}}{10000}+\frac{1}{10000}\\
\frac{1}{100} & -\frac{1}{1000} & \frac{w_{11}}{10000}+\frac{w_{12}}{10000}+\frac{1}{10000} & -\frac{w_{11}}{50000}-\frac{w_{12}}{50000}-\frac{1}{100000}\\
0 & \frac{1}{100} & -\frac{1}{1000} & \frac{w_{12}}{10000}+\frac{w_{22}}{10000}+\frac{1}{10000}\\
\frac{1}{100} & -\frac{1}{1000} & \frac{w_{12}}{10000}+\frac{w_{22}}{10000}+\frac{1}{10000} & -\frac{w_{12}}{50000}-\frac{w_{22}}{50000}-\frac{1}{100000}
\end{array}\right]}}
\]
\medskip{}
which has still $rank=4$. The interesting question now is how to
pick appropriate values for all the coefficients, so that the system
does indeed remain stable. Notice that if all the weights are the
same, the system becomes uncontrollable, as the $\Gamma's$ rank does
indeed drop. For instance, with $w_{11}=w_{12}=w_{21}=w_{22}=1$,
we obtain the following matrix:

$\Gamma^{''}=\left(\begin{array}{rrrr}
0 & \frac{1}{100} & -\frac{1}{1000} & \frac{3}{10000}\\
\frac{1}{100} & -\frac{1}{1000} & \frac{3}{10000} & -\frac{1}{20000}\\
0 & \frac{1}{100} & -\frac{1}{1000} & \frac{3}{10000}\\
\frac{1}{100} & -\frac{1}{1000} & \frac{3}{10000} & -\frac{1}{20000}
\end{array}\right)$whose rank is obviously 2, for the $3^{rd}$and $4^{th}$ rows just
repeat the first two rows. We can regain rank by setting weights to
be all different, of course, but also just by differentiating the
weights of the first unit from the weight of the second one. With
$w_{11}=w_{12}=1$ and $w_{21}=w_{22}=2$, we have the matrix

$\left(\begin{array}{rrrr}
0 & \frac{1}{100} & -\frac{1}{1000} & \frac{3}{10000}\\
\frac{1}{100} & -\frac{1}{1000} & \frac{3}{10000} & -\frac{1}{20000}\\
0 & \frac{1}{100} & -\frac{1}{1000} & \frac{1}{2500}\\
\frac{1}{100} & -\frac{1}{1000} & \frac{1}{2500} & -\frac{7}{100000}
\end{array}\right)$ whose rank is indeed 4.\medskip{}

But let us look at observability first.

\section{Observability}

Observability theorem (see ???):
\begin{thm}
A system is completely observable, iff $rank(\Omega)=n,$ where $n$
is the dimension of the state variable and $\Omega=\begin{bmatrix}C\\
CA\\
...\\
CA^{n-1}
\end{bmatrix}$ is the system's observability matrix.
\end{thm}
Computing the observability matrix for our 2 unit sample homeostat
we obtain the following matrix:

$\Omega=\left(\begin{array}{rrrr}
1 & 0 & 1 & 0\\
0 & 1 & 0 & 1\\
\frac{w_{11}}{m_{1}}+\frac{w_{12}}{m_{2}} & -\frac{v_{1}}{m_{1}} & \frac{w_{12}}{m_{1}}+\frac{w_{22}}{m_{2}} & -\frac{v_{2}}{m_{2}}\\
-\frac{v_{1}w_{11}}{m_{1}^{2}}-\frac{v_{2}w_{12}}{m_{2}^{2}} & \frac{v_{1}^{2}}{m_{1}^{2}}+\frac{w_{11}}{m_{1}}+\frac{w_{12}}{m_{2}} & -\frac{v_{1}w_{12}}{m_{1}^{2}}-\frac{v_{2}w_{22}}{m_{2}^{2}} & \frac{v_{2}^{2}}{m_{2}^{2}}+\frac{w_{12}}{m_{1}}+\frac{w_{22}}{m_{2}}
\end{array}\right)$

\medskip{}

whose rank is indeed 4, as it can be easily seen. Hence the system
is completely observable---assuming all the coefficient are indeed
different! With Homeo's default values, as above, the $\Omega$ matrix
looks like the following:

\begin{center}
$\Omega^{'}=\left[\begin{array}{rrrr}
1 & 0 & 1 & 0\\
0 & 1 & 0 & 1\\
\frac{w_{11}}{100}+\frac{w_{12}}{100} & -\frac{1}{10} & \frac{w_{12}}{100}+\frac{w_{22}}{100} & -\frac{1}{10}\\
-\frac{w_{11}}{1000}-\frac{w_{12}}{1000} & \frac{w_{11}}{100}+\frac{w_{12}}{100}+\frac{1}{100} & -\frac{w_{12}}{1000}-\frac{w_{22}}{1000} & \frac{w_{12}}{100}+\frac{w_{22}}{100}+\frac{1}{100}
\end{array}\right]$
\par\end{center}

TO BE CONTINUED....

\chapter{Describing a NN-controlled Braitenberg vehicle according to control
theory}

The goal is to get insights about the behavior of a Braitenberg vehicle
from its description according to control theory. In particular, the
eventual goal is to try to reverse engineer the NN controller so to
speak. That is: 
\begin{itemize}
\item Describe mathematically a Braitenberg vehicle (first Type 1, then,
eventually, Type 2) that achieves some predefined goal (for instance:
goal-seeking, or wandering) and then try to infer what the parameters
of the controller must be like in order for those those goals to be
achieved

The hope is that this description would produce get some insights
on the dynamic behavior of the vehicle and, hopefully, help determine
the parameters' range either for direct manipulation of the NN or
for GA's work.
\end{itemize}

\section{Preliminaries}

General points about this exercise:
\begin{itemize}
\item Start with type 1
\item Simplify vehicle's internal dynamics as much as possible. Indeed,
start with no dynamics (as in Braitenberg's work, in a certain sense)
\item Simplify input structure as much as possible
\end{itemize}

\section{From differential equations to state space form, in general}
\begin{enumerate}
\item Provide a description of the system as a differential equation describing
the temporal evolution of the system's state
\item The system with state $\mathbf{x}\in\mathbb{R}^{n}$ (the \emph{state
vector}), input $\mathbf{u}\in\mathbb{R}^{m}$ (the \emph{input vector}
representing the control variables ) and output $\mathbf{y}\in\mathbb{R}^{p}$
(the signal vector representing the measured signal) is represented
by 
\[
\frac{dx}{dt}=f(x,u),\,\,\,\,\,y=h(x,u)
\]

where $f:\mathbb{R}^{n}\times\mathbb{R}^{m}\rightarrow\mathbb{R}^{n}$and
$h:\mathbb{R}^{n}\times\mathbb{R}^{m}\rightarrow\mathbb{R}^{p}$ are
smooth mappings (Murray:34)
\item If the functions $f,h$ are linear in $x$ and $u$, the system is
called \emph{linear} and can be represented as:

\[
\dot{x}=Ax+Bu,\,\,\,\,y=Cx+Dx
\]

where $A:n\times n$ is the \emph{system matrix}, $B:n\times p$ is
the \emph{control matrix}, $C:p\times n$ is the \emph{sensor matrix
}and $D$ is the \emph{direct term} (frequently absent when the control
signal does not directly influence the system's output)
\item If the differential equation are not linear, linearize them around
an operating point (and look for possible problems involving linearizations)
\end{enumerate}

\section{A pure Braitenberg-1 vehicle with no NN and no physics}

Let us start with a pure Braitenberg Type-1 vehicle (both \emph{a}
and \emph{b} subtypes), with no physics of its own and no independent
controller, moving on a straight line. In fact, let us assume that
the whole environment is mono-dimensional. Let us also assume the
vehicle has a omnidirectional sensor receptive to a stimulus source
emitting a stimulus $S$ of intensity $I_{s}$ and located at $s$
on the line. 

Now, a Braitenberg vehicle of this kind, ignoring the system physics
for the time being, can be modeled by focusing on the state-dependent
system's output $y$ (the transformation of the outside world's input
to the system into a ``perceived'' output, i.e. a sensor reading)
the system's output's dependent actuator commands $u.$ The system's
state is simply its (global) position, hence \textbf{$\mathbf{x}=x$},
the input function $y(x)$ will be a function of the distance $\Vert s-x\Vert$
between the vehicle and the stimulus source and the stimulus's intensity
$I_{s}$. If the latter is inversely proportional to the distance
between the source and the robot, then $I_{s-x}=\frac{S}{1+\Vert s-x\Vert}$,
whereas $I_{s-x}=\frac{S}{1+\Vert s-x\Vert^{2}}$ if it is inversely
proportional to the square of the distance (as real lights would be)
and $y=f(I_{s-x})$ where $f$ may be a simple scaling parameter or
a more complex and possibly non linear function (as explicitly hinted
by Braitenberg himself.)

The input to the system, in the present case in which we have no NN
controller, would then be a simple multiplicative factor of the system's
output $u=Ky$ for vehicles of type 1\emph{b} with a direct connection
(the farther away, hence the smaller the stimulus, the weaker the
actuator) and $u=\frac{K}{y}$ for vehicles of type 1\emph{a} with
an inverse connection (the farther away, hence the smaller the stimulus,
the stronger the actuator)\emph{.} Thus, we can model a standard,
no physics, Type 1\emph{a }stimulus-avoider vehicle with a control
signal inversely proportional to the stimulus in an environment with
a linearly decreasing stimulus with the two equations: 
\begin{equation}
\dot{x}=u,\,\,\,y=\frac{\alpha S}{1+\Vert s-x\Vert},\,\,\,u=Ky\label{Type-1a}
\end{equation}

If the intensity $I_{s}$ is decreasing according to the inverse square
law, we would have:
\begin{equation}
\dot{x}=u,\,\,\,y=\frac{\alpha S}{1+\Vert s-x\Vert^{2}},\,\,\,u=Ky\label{eq:Type-1a-quadr}
\end{equation}

(we added a multiplicative constant $\alpha\in[0,1]$ indicating the
efficiency of the sensor in converting the actual intensity into a
perceived stimulus). For a control signal proportionally decreasing
with distance (the closer to the source, the weaker the stimulus)
modeling a stimulus-seeker, type 1\emph{b} vehicle the equations become:

\begin{equation}
\dot{x}=u,\,\,\,y=\frac{\alpha S}{1+\Vert s-x\Vert},\,\,\,u=\frac{K}{y},\,\textrm{or}\,\,\dot{x}=u,\,\,\,y=\frac{\alpha S}{1+\Vert s-x\Vert^{2}},\,\,\,u=\frac{K}{y}\label{eq:Type-1-b}
\end{equation}

As these equations model a Type 1 vehicle with omni-directional sensors,
they can also be assumed to provide a simplified model of a vehicle
with a skirt of sensors, in which case the function $y$ could be
interpreted as the (possibly weighted) average of the various sensors
along the skirts. 

\subsection{Directional sensors}

Let us now add the limitation that the sensor pointing is forward
and has a field of view $180^{\circ}$. In other words the sensor
does not see any stimulus behind the (point-like) robot. With this
limitation, the stimulus will be received only if the distance $s-x$
is positive (source in front of the robot) and zero otherwise. We
approximate the required behavior by multiplying the perceived intensity
by the logistic function scaled to the distance $(s-x)$: $\frac{1}{1+e^{\beta(s-x)}}$,
which will have value 0 whenever the distance $(s-x)$ is negative
and whose transition from 1 to 0 is sufficiently steep with a sufficiently
small negative factor $\beta$. Our function $y$ will thus become
$y=\frac{\alpha S}{1+\Vert s-x\Vert^{(2)}}*\frac{1}{1+e^{\beta(s-x)}}$,
which can be slightly simplified to $y=\frac{\alpha S}{1+\Vert s-x\Vert^{(2)}+e^{\beta(s-x)}}$.%
\begin{lyxgreyedout}
Notice that this expression is not general at all and it applies only
to the 1-D case. I will have to come up with a better expression for
the 2D case, in which we have sensors with arbitrary fields of view
$\theta$ located on a robot with heading $\rho$ (while the source
can be anywhere in the plane.%
\end{lyxgreyedout}

Alternatively, we cold use a piece-wise function for $y$ defined
as above if the angle $\theta$ between the robot's heading and the
source's location falls within an angle $\phi$ representing the range
of the sensor, and zero otherwise. Assuming sensors' bilateral symmetry:

\[
y=\begin{cases}
\frac{\alpha S}{1+\Vert s-x\Vert^{(2)}} & -\frac{\phi}{2}\le\theta\le\frac{\phi}{2}\\
0 & otherwise
\end{cases}
\]

In our 1-dimensional case, with $\phi=\pi$, and the sign of $cos\,\theta$
equal to the the sign of $(s-x)$, we have the equivalent formulation

\[
y=\begin{cases}
\frac{\alpha S}{1+\Vert s-x\Vert^{(2)}} & (s-x)\ge0\\
0 & (s-x)<0
\end{cases}
\]

for a forward-looking sensor, with conditions switched for a rear-looking
one. %
\begin{lyxgreyedout}
The problem with this solution is that the $y$ is now not only non-linear
(as it was the case for the logistic function), but not even differentiable
at $(s-x)=0$, which represents the only fixed point of the system
considered so far. Hence the existence/uniqueness of the solution
may be problematic (see Strogatz 27). %
\end{lyxgreyedout}


\subsection{Qualitative analysis}

For a qualitative analysis, it is useful to express the equations
as a single differential equation of $\dot{x},$ thereby obtaining
the following for the system of Type 1\emph{a} with omnidirectional
sensor:

\begin{equation}
\dot{x}=\frac{\alpha KS}{1+\Vert s-x\Vert^{(2)}}\label{eq:Type1-a-system}
\end{equation}

and respectively, for Type 1\emph{b }with omnidirectional sensor
\begin{equation}
\dot{x}=\frac{K(1+\Vert s-x\Vert^{(2)})}{\alpha S}\label{eq:Type1-b-system}
\end{equation}
where the parenthesis around the quadratic term indicates that the
stimulus-decreasing function can be linear or quadratic. 

A qualitative analysis in terms of phase portrait of the systems represented
by equations \eqref{fig:Type-1a-and1b-phase-omnidir-portraits} and
\eqref{eq:Type1-b-system} is shown in figure \eqref{fig:Type-1a-and1b-phase-omnidir-portraits}.
It is obvious that Type 1b as an unstable fixed point as $s$, whereas
Type 1a has no fixed points at all. 

\begin{figure}
\subfloat[Type 1\emph{a}, ``the closer the stronger'']{

\includegraphics[scale=0.7]{Images/Braiten-1a-omni-phase-portrait}

}\subfloat[Type 1\emph{b}, ``the closer, the weaker'']{

\includegraphics[scale=0.7]{Images/Braiten-1b-omni-phase-portrait}

}\caption{\label{fig:Type-1a-and1b-phase-omnidir-portraits}Omnidirectional
sensors Type 1a and Type 1b phase portraits with a stimulus source
at location $s=5.$ All other relevant parameters $(\alpha,S,K)$
are equal to 1. }

\end{figure}

With the forward-looking sensors, instead, we would have: 

\begin{equation}
\dot{x}=\frac{\alpha KS}{1+\Vert s-x\Vert^{(2)}+e^{\beta(s-x)}}\label{eq:Type1-a-system-1}
\end{equation}

and respectively, for Type 1\emph{b }with forward-looking sensor
\begin{equation}
\dot{x}=\frac{K(1+\Vert s-x\Vert^{(2)}}{\alpha S}*\frac{1}{1+e^{\beta(s-x)}}=\frac{K(1+\Vert s-x\Vert^{(2)})}{\alpha S(1+e^{\beta(s-x)})}\label{eq:Type1-b-system-1}
\end{equation}

The phase portraits of these two systems are illustrated in figure
\ref{fig:Type-1a-and1b-phase-directional-portraits}. 
\begin{figure}

\subfloat[Type 1\emph{a}, ``the closer the stronger'' ]{

\includegraphics[scale=0.7]{Images/Braiten-1a-phase-portrait}

}\subfloat[Type 1\emph{b}, ``the closer, the weaker'']{

\includegraphics[scale=0.7]{Images/Braiten-1b-phase-portrait}}\caption{\label{fig:Type-1a-and1b-phase-directional-portraits}Type 1a and
Type 1b phase portraits with a stimulus source at location $s=5.$
All other relevant parameters ($\alpha,S,K,$) are equal to 1. The
logistic function factor $\beta=-100$.}

\end{figure}

The phase portraits show that in either case there is a fixed point
at $x_{0}=s$, and indeed all points to the right, ($x_{0}>s)$ are
fixed points as well.  In other words, both vehicles are stimulus-seekers,
whit the difference that Type 1a rushes to the source (it is ``aggressive,''
Braitenberg would say) and the Type 1b, on the contrary, starts very
fast and slows down as it approaches the source (it is ``timid'').

\section{Still to do from here on}

\subsection{Find a better and more general solution for directionality of sensors,
extensible to 2D case}

\subsection{Multimodal inputs and multi-inputs}

\subsection{Connection of NN to system}

Looking ahead, a NN-controlled Type-1 vehicle would replace the expressions
for $y$ and $u$ with more complex functions. In fact, it would replace
them with a whole, possibly recurrent network with input \emph{$I_{s}$}
and output $u$.

However, before we get to NN-controlled vehicles, we need to investigate
a conversion into state form. The equations in \eqref{Type-1a}, \eqref{eq:Type-1a-quadr},
and \eqref{eq:Type-1-b}, are non-linear, thereby a linearization
is necessary and we need to decide around which points to linearize.%
\begin{lyxgreyedout}
I NEED TO CHECK: HOSANHA AS WELL AS MURRAY/ASTROM ON LINEARIZATION,
THE BOOK ON FEEDBACK CONTROL OF DYNAMIC SYSTEM FOR NON-LINEAR SYSTEMS,
RANO (THE GUY WHO DID THE MODEL FOR TYPE 2) FOR EXAMPLES, AND BEER
FOR THE ANALYSIS OF CTRNN%
\end{lyxgreyedout}
{} 


\section{A Braitenberg-1 with a NN controller}

\section{A Braitenberg-1 with a Homeo controller}

\section{A Braitenberg-1 with physics and no NN controller}

Assuming a more realistic physical setup and a second order differential
equation governing the motion of the Type 1 vehicle, the differential
equations above (for Type 1\emph{a} and Type 1\emph{b}) now become,
respectively:
\[
\ddot{x}=\frac{kS}{1+\Vert s-x\Vert^{2}}-V\dot{x}\,\,\,,\,\,\,\ddot{x}=\Vert s-x\Vert^{2}*kS-V\dot{x}
\]

where $V$ is a constant indicating the velocity-proportional friction
coefficient. In this case, our state becomes $x=\begin{bmatrix}x_{1}\\
x_{2}
\end{bmatrix}=\begin{bmatrix}x\\
\dot{x}
\end{bmatrix}$ and the linearization of the system involves computing the Jacobian,
which is...

CONTINUE HERE
\end{document}
